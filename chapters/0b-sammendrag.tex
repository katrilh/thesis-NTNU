\chapter*{Sammendrag}

I takt med en økning av enheter som kan registrere og lagre bevegelsesdata har interessen for å analysere denne dataen vokst. 
Å begrene likhetsavstand mellom observasjoner er avgjørende for å kunne gjennomføre all type analyse. 
For tidsrekkedata finnes det mange ulike algoritmer som måler deres likhetsavstand 

Denne oppgaven består av teoretiske og eksperimentelle undersøkelser av syv likhetsavstandsmål.
Den teoretiske delen greier ut om ulike mål av sporlikhet, og hvordan disse er inspirert av dataformatet til tidsrekker.
Den eksperimentelle delen av oppgaven handler om  hvordan teorien utspiller seg i praksis.
Vi analyserer et datasett bestående  av tilnærmet vilkårlige spor. Analysen tydeliggjør hvordan de ulike algoritmene bruker forskjellige definisjoner av likhet.
