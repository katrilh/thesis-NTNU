\chapter{Introduction}
\label{ch:1}

\section{Motivation}
% With the rapid development of technology that can be used to record and store location data, the interest of examining this data has grown in tandem. The [places] where location data is interesting are vast, anything that moves is subject to trajectory analysis. We have seen this technology being put to use in everything from large scale projects such as vessels (sailing) across the globe to the [tinier] movement or mapping mobility at individual intersections. 

% By convention, the mobility data is represented as timestamped sequences of locations, and  are referred to as trajectories. A fundamental operator in trajectory analysis is the similarity operator. The subfield trajectory mining is fully reliant on being able to numerically quantify how similar two trajectories are [69].


Several modern devices can collect location data which is usually stored as trajectories.
With the rise of available data, the interest in analyzing it has been increasing as well. 
In order to achieve a meaningful analysis, we depend on the similarity operator, but challenges arise when narrowing now what it means for trajectories to be similar. 

In computer vision, object outlines can be mapped as trajectories and thus their shape similarity becomes essential for recognition. 
Furthermore, the ability to determine the similarity between trajectories is essential to trajectory database management. Pattern mining, classification, outlier detection, observational uncertainty, and forecasts are examples of queries that depend on a similarity measure\cite{31-ShapebasedSimilarity,80-TrajectoryData,92-TrajectoryIndexing,51-HausdorffDistance}.

Trajectories have a compound data format; they are a series of observations that vary over time and location. The location data may itself be multi-dimensional. Complex geometric shapes do not have a simple notion of shortest distance. This gives rise to different interpretations of how to quantify trajectory similarity and in turn the development of different algorithms. 

In this thesis, seven different similarity measures are used to perform a comparative study. 
In contrast to the majority of existing work, we do not seek an ordered ranking of the most efficient or accurate measures. 
Rather, the aim is to accurately describe the properties of the selected measures discussed and guide someone to pick the most appropriate measure of them for their applications. 


% With the aim of visualizing which trajectory features any of the selected measures emphasize, we run two clustering tasks. 
% We underline that the clustered generated by the measures is meant to be illustrative. Trajectory clustering, while being closely related to trajectory similarity, is a separate area of research. 




% Problem Statement and

\clearpage
\section{Research Objectives} \label{1_ro}
Given the purpose of this thesis, we formulate the research objectives as the following:   



\textbf{Research Objective 1: }
Describe qualities and traits of both conventional and newer similarity measures.

\textbf{Research Objective 2: }
Determine how different similarity algorithms treat trajectory features and display these contrasts visually.

\textbf{Research Objective 3: }
Shed light on how similarity algorithm selection matters for a specific application.



% Examine how the notion trajectory similarity changes with different algorithms
% % from said results and  discuss possible reasons behind the end result. 

% \textbf{Research Objective 3: }
% Examine what kinds of experiments have been done previously to compare similarity measures.

% \textbf{RO 4: }
% (Numerically and graphically portray results from similarity measure analysis.) 

% \textbf{RO 4: }
% Reason similarity measures from said results and  discuss possible reasons behind the end result. 

% \subsection*{Scope Limitation}
% Due to time limitations, we have constrains that scope of this theis in the following ways:

% \begin{itemize}
% \item  we shall work with a data set of about 500 entries 
% \item We truncate all trajectories to have the same length so that we may simplify the measures similarly scores computation 
% \end{itemize}
% % sooo, do i movethis?

\section{Thesis Outline}

% https://docs.google.com/document/d/1NKcKltS1OYkH7EfAXRZtQ_0yF7l4gFiZmFWvbePRgFw/edit#
The rest of the thesis is structured as follows:


\textbf{\Cref{ch:2}} presents concepts and notations that will be used throughout the thesis. 
We narrow down the definition of a trajectory, discuss how to concretize similarity between two trajectories, and review trajectory mining as a comparative technique for trajectory similarity. 

\textbf{\Cref{ch:3}} pitches seven trajectory similarity algorithms. We describe how they quantify alikeness and bring up their associated advantages and shortcomings.

\textbf{\Cref{ch:4}} briefly recounts the contents and conclusions of existing comparative studies of similarity algorithms. 
%We elaborate on  the uniqueness of this  work 

\textbf{\Cref{ch:5}} proceeds to define the implementation details of the experiments. 
It also describes the methodology, how we evaluated the results, and discloses the simplifications that were made. 


\textbf{\Cref{ch:6}} is a presentation of the results. 
They are presented in three parts: the similarity scores, the results from the mining tasks, and evaluation of those results. 

\textbf{\Cref{ch:7}} discusses the results that were presented in the preceding chapter. 
We emphasize what each similarity measure's result was like in comparison to its theoretical characterization. % We also include some potential applications for the work that was done.  

\textbf{\Cref{ch:8}} contains the concluding remarks for this thesis. We reflect on what has been achieved and revisit the research objectives. 
Finally, we discuss how this set up could be advanced for future work. 











